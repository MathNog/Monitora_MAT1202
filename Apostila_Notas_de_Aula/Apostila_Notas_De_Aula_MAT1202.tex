\documentclass[12pt]{article}

\usepackage[utf8]{inputenc}
\usepackage[brazil]{babel}
\usepackage[a4paper,left=3cm, right=2cm,top=3cm, bottom=2cm]{geometry}
\usepackage{amsmath}
\usepackage{graphicx}
\usepackage{float}
\usepackage{authblk}
\usepackage{fancyhdr}
\usepackage{xcolor}

\title{\textbf{Monitoria MAT1202 - Álgebra Linear 2 \\Apostila Notas de Aula }}

\author[]{\textbf{Matheus Nogueira}}

\date{}
\pagestyle{fancy}
\fancyhf{}
\lhead{{\small \textcolor{gray}{Apostila Monitoria MAT1202}}}
\renewcommand{\headrulewidth}{0pt}
\fancyfoot[C]{\thepage}
\begin{document}
\maketitle
\begin{abstract}
	Este documento consiste nas notas de aula da monitoria de MAT1202. Este material foi produzido com base em minhas anotações do curso de Álgebra Linear 2 do semestre 20.2 e do livro \textit{Álgebra Linear e suas aplicações}, de \textit{Gilbert Strang}. Qualquer dúvida, favor entrar em contato \textit{matnogueira@gmail.com}
\end{abstract}
\tableofcontents
\pagebreak
\section{Sistemas Lineares e Eliminação Gaussiana}
\subsection{Sistemas Lineares e Notação Matricial}

Nosso foco é estudar sistemas de equações da forma $Ax=b$, onde $A$ é a matriz com os termos que acompanham as variáveis (incógnitas), $x$ é o vetor coluna com as incógnitas e $b$ é o vetor coluna com os termos independentes.\\

\textbf{Exemplo:} Seja o seguinte sistema de equações...

\begin{align*}
	x+2y+3&=2\\-x+y-z&=-3\\2x+y-z&=0
\end{align*}

Escrevê-lo em forma matricial é definir as seguinte matriz e vetores:
\begin{equation*}
	A=
	\begin{bmatrix}
		1 & 2 & 3\\
		-1 & 1 & -1\\
		2 & 1 & -1
	\end{bmatrix} \mbox{, } 
	x=
	\begin{bmatrix}
		x\\ y\\z
	\end{bmatrix} \mbox{ e } 
	b=
	\begin{bmatrix}
		2 \\ -3\\0
	\end{bmatrix}
\end{equation*}

Não é difícil perceber que a multiplicação representada por $Ax$ resulta exatamente no sistema linear inicial.

\subsection{Solução de um Sistema Linear}

Nossa estratégia para calcular a solução de um sistema de equações lineares será a \textbf{Eliminação Gaussiana}. 

Este método consiste em realizar operações na matriz do sistema $Ax=b$, chamadas \textit{operações elementares}, para chegar a um \textit{sistema triangular}. Ao ser obtido este sistema, basta realizar uma série de substituições retroativas para chegar à solução.\\

\textbf{Definição:} Matrizes Triangulares

Uma matriz é triangular - superior ou inferior - se todas as entradas abaixo ou acima, respectivamente, da diagonal principal são nulas. A matriz $A$ abaixo é triangular superior, enquanto que $B$ é triangular inferior.

\begin{equation*}
	A=
	\begin{bmatrix}
		1 & 2 & 3\\
		0 & 1 & -1\\
		0 & 0 & -1
	\end{bmatrix} \mbox{, } 
	B=
	\begin{bmatrix}
		1 & 0 &0\\
		-1 & 1 & 0\\
		2 & 1 & -1
	\end{bmatrix}
\end{equation*}

São 3 os possíveis tipos de solução de um sistema linear:
\begin{enumerate}
	\item Exatamente 1 solução
	\item Infinitas Soluções
	\item Não há solução
\end{enumerate}

\textbf{Observação:} lembrem-se que, para verificar qual das opções acima é a o caso da matriz a ser estudada, podemos olhar para o \textit{determinante} da matriz. Se seu valor for zero, o sistema possui infinitas soluções ou nenhuma solução. Se for diferente de zero, uma solução.

\subsubsection{Operações Elementares}

\textbf{Definição:}
dado um sistema linear $Ax=b$, são 3 as operações elementares que não alteram a solução do sistema.
\begin{enumerate}
	\item Permutação de linhas ($L_i\leftrightarrow L_j$)
	\item Multiplicação de linha por escalar ($L_i\rightarrow L_i \cdot k,k \neq 0$)
	\item Somar um múltiplo de uma linha a outra linha ($L_i \rightarrow L_i+k\cdot L_j$)
\end{enumerate}

\subsubsection{Matrizes das operações elementares}

Veremos que cada uma das 3 operações elementares descritas pode ser representada por meio de matrizes da seguinte forma:\\

Se queremos realizar a operação elementar $e$ sobre a matriz $A$, devemos realizar a multiplicação $E\cdot A$, onde $E$ é a matriz que representa a operação elementar $e$. \\

Vejamos as como montar as matrizes para as mesmas 3 operações já apresentadas. Por facilidade, usaremos matrizes $3x3$, pois o raciocínio para outras dimensões é o mesmo. Começamos sempre com a matriz identidade e:
\begin{enumerate}
	\item Permutação de linhas ($L_i\leftrightarrow L_j$):\\ basta permutar as linhas da matriz identidade de acordo com as linhas a serem permutadas na matriz A
	\item Multiplicação de linha por escalar ($L_i\rightarrow L_i \cdot k,k \neq 0$):\\ multiplicamos a linha correspondente da matriz identidade pelo escalar em questão.
	\item Somar um múltiplo de uma linha a outra linha ($L_i \rightarrow L_i+k\cdot L_j$):\\ colocamos na entrada $i,j$ da matriz identidade o valor de $k$ com o devido sinal.
\end{enumerate}

\begin{align}
	L_2\leftrightarrow L_3 \implies &
	\begin{bmatrix}
		1 & 0 &0\\
		0 & 1 & 0\\
		0 & 0 & 1
	\end{bmatrix} \leftrightarrow 
	\begin{bmatrix}
		1 & 0 &0\\
		0 & 0 & 1\\
		0 & 1 & 0
	\end{bmatrix}=E\\
	L_2\rightarrow L_2 \cdot k\implies &
	\begin{bmatrix}
		1 & 0 &0\\
		0 & 1 & 0\\
		0 & 0 & 1
	\end{bmatrix} \leftrightarrow 
	\begin{bmatrix}
		1 & 0 &0\\
		0 & k\cdot 1 & 0\\
		0 & 0 & 1
	\end{bmatrix}=E\\
	L_3 \rightarrow L_3-2\cdot L_1 \implies &
	\begin{bmatrix}
		1 & 0 &0\\
		0 & 1 & 0\\
		0 & 0 & 1
	\end{bmatrix} \leftrightarrow 
	\begin{bmatrix}
		1 & 0 &0\\
		0 & 1 & 0\\
		-2 & 0 & 1
	\end{bmatrix}=E
\end{align}\\

Ao final da \textit{Eliminação Gaussiana}, depois de serem realizadas todas as devidas \textit{operações elementares}, a matriz obtida estará na forma \textbf{escalonada}, isto é:
\begin{enumerate}
	\item Se existem linhas nulas elas devem ser as últimas da matriz.
	\item Em quaisquer duas linhas sucessivas não nulas, o pivô (primeiro elemento não nulo) da linha inferior deve estar mais à direita que o da linha superior.
	\item Abaixo do pivô todas as entradas são nulas.
\end{enumerate}

\subsection{Exemplo}

Calculemos a solução do seguinte sistema, mostrando as matrizes das operações elementares.

\begin{align*}
	2x+y+z&=5\\4x-6y&=2\\-2x+7y+2z&=9
\end{align*}\\

Em forma matricial o sistema é: \\

\begin{equation*}
	\begin{bmatrix}
		2 & 1 & 1\\
		4 & -6 & 0\\
		-2 & 7 & 2
	\end{bmatrix} \cdot
	\begin{bmatrix}
		x\\ y\\z
	\end{bmatrix} =
	\begin{bmatrix}
		5 \\ -2\\9
	\end{bmatrix}
\end{equation*}


Seja a matriz aumentada a ser escalonada a seguir: 
\begin{equation*}
	\begin{bmatrix}
		2 & 1 & 1 & 5\\
		4 & -6 & 0 & -2\\
		-2 & 7 & 2 & 9
	\end{bmatrix}
\end{equation*}

Comecemos as operações elementares para chegar à matriz escalonada. A cada operação, indicaremos a matriz $E$ correspondente.

\begin{equation*}
	L_2 \rightarrow L_2 -2L_1 \mbox{ sendo } 
	E_1=\begin{bmatrix}
		1 & 0 &0\\
		-2 & 1 & 0\\
		0 & 0 & 1
	\end{bmatrix}\\
\end{equation*}
Nosso sistema fica...
\begin{equation*}
	\begin{bmatrix}
		2 & 1 & 1 & 5\\
		0 & -8 & -2 & -12\\
		-2 & 7 & 2 & 9
	\end{bmatrix}
\end{equation*}\\


\begin{equation*}
	L_3 \rightarrow L_3 +L_1 \mbox{ sendo } 
	E_2=\begin{bmatrix}
		1 & 0 &0\\
		0 & 1 & 0\\
		1 & 0 & 1
	\end{bmatrix}\\
\end{equation*}
Nosso sistema fica...
\begin{equation*}
	\begin{bmatrix}
		2 & 1 & 1 & 5\\
		0 & -8 & -2 & -12\\
		0 & 8 & 3 & 14
	\end{bmatrix}
\end{equation*}


\begin{equation*}
	L_3 \rightarrow L_3 +L_2 \mbox{ sendo } 
	E_3=\begin{bmatrix}
		1 & 0 &0\\
		0 & 1 & 0\\
		0 & 1 & 1
	\end{bmatrix}\\
\end{equation*}
Nosso sistema fica...
\begin{equation*}
	\begin{bmatrix}
		2 & 1 & 1 & 5\\
		0 & -8 & -2 & -12\\
		0 & 0 & 1 & 2
	\end{bmatrix}
\end{equation*}

Chegamos à matriz escalonada. Agora basta realizar algumas substituições retroativas para calcularmos a solução.

Lendo e substituindo o sistema de baixo para cima temos:

\begin{align*}
	z&=2 \\
	-8y-2(2)&=-12 \rightarrow y=1 \\
	2x+1+2&=5 \rightarrow x=1	
\end{align*}

Note que chegamos a uma solução única, o que faz sentido pois $\det(A)=-16\neq0$

Utilizando as matrizes das operações elementares, chegaríamos na mesma matriz escalonada:

\begin{align*}
	E_3\cdot E_2 \cdot E_1 \cdot A &\mbox{ , onde A é a matriz aumentada} \\
	\begin{bmatrix}
		1 & 0 &0\\
		0 & 1 & 0\\
		0 & 1 & 1
	\end{bmatrix}\cdot
	\begin{bmatrix}
		1 & 0 &0\\
		0 & 1 & 0\\
		1 & 0 & 1
	\end{bmatrix} \cdot
	\begin{bmatrix}
		1 & 0 &0\\
		-2 & 1 & 0\\
		0 & 0 & 1
	\end{bmatrix} \cdot&
	\begin{bmatrix}
		2 & 1 & 1 & 5\\
		4 & -6 & 0 & -2\\
		-2 & 7 & 2 & 9
	\end{bmatrix} = 
	\begin{bmatrix}
		2 & 1 & 1 & 5\\
		0 & -8 & -2 & -12\\
		0 & 0 & 1 & 2
	\end{bmatrix}
\end{align*}

\subsection{Conclusão}
Com este material sabemos como encontrar a solução de um sistema linear utilizando a Eliminação Gaussiana e as operações Elementares, com suas respectivas matrizes. O próximo assunto a ser abordado será \textbf{Fatoração LU}.
\end{document}

