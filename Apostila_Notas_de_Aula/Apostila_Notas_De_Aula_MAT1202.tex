\documentclass[12pt]{article}

\usepackage[utf8]{inputenc}
\usepackage[brazil]{babel}
\usepackage[a4paper,left=3cm, right=2cm,top=3cm, bottom=2cm]{geometry}
\usepackage{amsmath}
\usepackage{graphicx}
\usepackage{float}
\usepackage{authblk}
\usepackage{fancyhdr}
\usepackage{xcolor}

\title{\textbf{Monitoria MAT1202 - Álgebra Linear 2 \\Apostila Notas de Aula }}

\author[]{\textbf{Matheus Nogueira}}

\date{}
\pagestyle{fancy}
\fancyhf{}
\lhead{{\small \textcolor{gray}{Apostila Monitoria MAT1202}}}
\renewcommand{\headrulewidth}{0pt}
\fancyfoot[C]{\thepage}
\begin{document}
\maketitle
\begin{abstract}
	Este documento consiste nas notas de aula da monitoria de MAT1202. Este material foi produzido com base em minhas anotações do curso de Álgebra Linear 2 do semestre 20.2 e do livro \textit{Álgebra Linear e suas aplicações}, de \textit{Gilbert Strang}. Qualquer dúvida, favor entrar em contato \textit{matnogueira@gmail.com}
\end{abstract}
\tableofcontents
\pagebreak
\section{Sistemas Lineares e Eliminação Gaussiana}
\subsection{Sistemas Lineares e Notação Matricial}

Nosso foco é estudar sistemas de equações da forma $Ax=b$, onde $A$ é a matriz com os termos que acompanham as variáveis (incógnitas), $x$ é o vetor coluna com as incógnitas e $b$ é o vetor coluna com os termos independentes.\\

\textbf{Exemplo:} Seja o seguinte sistema de equações...

\begin{align*}
	x+2y+3&=2\\-x+y-z&=-3\\2x+y-z&=0
\end{align*}

Escrevê-lo em forma matricial é definir as seguinte matriz e vetores:
\begin{equation*}
	A=
	\begin{bmatrix}
		1 & 2 & 3\\
		-1 & 1 & -1\\
		2 & 1 & -1
	\end{bmatrix} \mbox{, } 
	x=
	\begin{bmatrix}
		x\\ y\\z
	\end{bmatrix} \mbox{ e } 
	b=
	\begin{bmatrix}
		2 \\ -3\\0
	\end{bmatrix}
\end{equation*}

Não é difícil perceber que a multiplicação representada por $Ax$ resulta exatamente no sistema linear inicial.

\subsection{Solução de um Sistema Linear}

Nossa estratégia para calcular a solução de um sistema de equações lineares será a \textbf{Eliminação Gaussiana}. 

Este método consiste em realizar operações na matriz do sistema $Ax=b$, chamadas \textit{operações elementares}, para chegar a um \textit{sistema triangular}. Ao ser obtido este sistema, basta realizar uma série de substituições retroativas para chegar à solução.\\

\textbf{Definição:} Matrizes Triangulares

Uma matriz é triangular - superior ou inferior - se todas as entradas abaixo ou acima, respectivamente, da diagonal principal são nulas. A matriz $A$ abaixo é triangular superior, enquanto que $B$ é triangular inferior.

\begin{equation*}
	A=
	\begin{bmatrix}
		1 & 2 & 3\\
		0 & 1 & -1\\
		0 & 0 & -1
	\end{bmatrix} \mbox{, } 
	B=
	\begin{bmatrix}
		1 & 0 &0\\
		-1 & 1 & 0\\
		2 & 1 & -1
	\end{bmatrix}
\end{equation*}

São 3 os possíveis tipos de solução de um sistema linear:
\begin{enumerate}
	\item Exatamente 1 solução
	\item Infinitas Soluções
	\item Não há solução
\end{enumerate}

\textbf{Observação:} lembrem-se que, para verificar qual das opções acima é a o caso da matriz a ser estudada, podemos olhar para o \textit{determinante} da matriz. Se seu valor for zero, o sistema possui infinitas soluções ou nenhuma solução. Se for diferente de zero, uma solução.

\subsubsection{Operações Elementares}

\textbf{Definição:}
dado um sistema linear $Ax=b$, são 3 as operações elementares que não alteram a solução do sistema.
\begin{enumerate}
	\item Permutação de linhas ($L_i\leftrightarrow L_j$)
	\item Multiplicação de linha por escalar ($L_i\rightarrow L_i \cdot k,k \neq 0$)
	\item Somar um múltiplo de uma linha a outra linha ($L_i \rightarrow L_i+k\cdot L_j$)
\end{enumerate}

\subsubsection{Matrizes das operações elementares}

Veremos que cada uma das 3 operações elementares descritas pode ser representada por meio de matrizes da seguinte forma:\\

Se queremos realizar a operação elementar $e$ sobre a matriz $A$, devemos realizar a multiplicação $E\cdot A$, onde $E$ é a matriz que representa a operação elementar $e$. \\

Vejamos as como montar as matrizes para as mesmas 3 operações já apresentadas. Por facilidade, usaremos matrizes $3x3$, pois o raciocínio para outras dimensões é o mesmo. Começamos sempre com a matriz identidade e:
\begin{enumerate}
	\item Permutação de linhas ($L_i\leftrightarrow L_j$):\\ basta permutar as linhas da matriz identidade de acordo com as linhas a serem permutadas na matriz A
	\item Multiplicação de linha por escalar ($L_i\rightarrow L_i \cdot k,k \neq 0$):\\ multiplicamos a linha correspondente da matriz identidade pelo escalar em questão.
	\item Somar um múltiplo de uma linha a outra linha ($L_i \rightarrow L_i+k\cdot L_j$):\\ colocamos na entrada $i,j$ da matriz identidade o valor de $k$ com o devido sinal.
\end{enumerate}

\begin{align}
	L_2\leftrightarrow L_3 \implies &
	\begin{bmatrix}
		1 & 0 &0\\
		0 & 1 & 0\\
		0 & 0 & 1
	\end{bmatrix} \leftrightarrow 
	\begin{bmatrix}
		1 & 0 &0\\
		0 & 0 & 1\\
		0 & 1 & 0
	\end{bmatrix}=E\\
	L_2\rightarrow L_2 \cdot k\implies &
	\begin{bmatrix}
		1 & 0 &0\\
		0 & 1 & 0\\
		0 & 0 & 1
	\end{bmatrix} \leftrightarrow 
	\begin{bmatrix}
		1 & 0 &0\\
		0 & k\cdot 1 & 0\\
		0 & 0 & 1
	\end{bmatrix}=E\\
	L_3 \rightarrow L_3-2\cdot L_1 \implies &
	\begin{bmatrix}
		1 & 0 &0\\
		0 & 1 & 0\\
		0 & 0 & 1
	\end{bmatrix} \leftrightarrow 
	\begin{bmatrix}
		1 & 0 &0\\
		0 & 1 & 0\\
		-2 & 0 & 1
	\end{bmatrix}=E
\end{align}\\

Ao final da \textit{Eliminação Gaussiana}, depois de serem realizadas todas as devidas \textit{operações elementares}, a matriz obtida estará na forma \textbf{escalonada}, isto é:
\begin{enumerate}
	\item Se existem linhas nulas elas devem ser as últimas da matriz.
	\item Em quaisquer duas linhas sucessivas não nulas, o pivô (primeiro elemento não nulo) da linha inferior deve estar mais à direita que o da linha superior.
	\item Abaixo do pivô todas as entradas são nulas.
\end{enumerate}

\subsection{Exemplo}

Calculemos a solução do seguinte sistema, mostrando as matrizes das operações elementares.

\begin{align*}
	2x+y+z&=5\\4x-6y&=2\\-2x+7y+2z&=9
\end{align*}\\

Em forma matricial o sistema é: \\

\begin{equation*}
	\begin{bmatrix}
		2 & 1 & 1\\
		4 & -6 & 0\\
		-2 & 7 & 2
	\end{bmatrix} \cdot
	\begin{bmatrix}
		x\\ y\\z
	\end{bmatrix} =
	\begin{bmatrix}
		5 \\ -2\\9
	\end{bmatrix}
\end{equation*}


Seja a matriz aumentada a ser escalonada a seguir: 
\begin{equation*}
	\begin{bmatrix}
		2 & 1 & 1 & 5\\
		4 & -6 & 0 & -2\\
		-2 & 7 & 2 & 9
	\end{bmatrix}
\end{equation*}

Comecemos as operações elementares para chegar à matriz escalonada. A cada operação, indicaremos a matriz $E$ correspondente.

\begin{equation*}
	L_2 \rightarrow L_2 -2L_1 \mbox{ sendo } 
	E_1=\begin{bmatrix}
		1 & 0 &0\\
		-2 & 1 & 0\\
		0 & 0 & 1
	\end{bmatrix}\\
\end{equation*}
Nosso sistema fica...
\begin{equation*}
	\begin{bmatrix}
		2 & 1 & 1 & 5\\
		0 & -8 & -2 & -12\\
		-2 & 7 & 2 & 9
	\end{bmatrix}
\end{equation*}\\


\begin{equation*}
	L_3 \rightarrow L_3 +L_1 \mbox{ sendo } 
	E_2=\begin{bmatrix}
		1 & 0 &0\\
		0 & 1 & 0\\
		1 & 0 & 1
	\end{bmatrix}\\
\end{equation*}
Nosso sistema fica...
\begin{equation*}
	\begin{bmatrix}
		2 & 1 & 1 & 5\\
		0 & -8 & -2 & -12\\
		0 & 8 & 3 & 14
	\end{bmatrix}
\end{equation*}


\begin{equation*}
	L_3 \rightarrow L_3 +L_2 \mbox{ sendo } 
	E_3=\begin{bmatrix}
		1 & 0 &0\\
		0 & 1 & 0\\
		0 & 1 & 1
	\end{bmatrix}\\
\end{equation*}
Nosso sistema fica...
\begin{equation*}
	\begin{bmatrix}
		2 & 1 & 1 & 5\\
		0 & -8 & -2 & -12\\
		0 & 0 & 1 & 2
	\end{bmatrix}
\end{equation*}

Chegamos à matriz escalonada. Agora basta realizar algumas substituições retroativas para calcularmos a solução.

Lendo e substituindo o sistema de baixo para cima temos:

\begin{align*}
	z&=2 \\
	-8y-2(2)&=-12 \rightarrow y=1 \\
	2x+1+2&=5 \rightarrow x=1	
\end{align*}

Note que chegamos a uma solução única, o que faz sentido pois $\det(A)=-16\neq0$

Utilizando as matrizes das operações elementares, chegaríamos na mesma matriz escalonada:

\begin{align*}
	E_3\cdot E_2 \cdot E_1 \cdot A &\mbox{ , onde A é a matriz aumentada} \\
	\begin{bmatrix}
		1 & 0 &0\\
		0 & 1 & 0\\
		0 & 1 & 1
	\end{bmatrix}\cdot
	\begin{bmatrix}
		1 & 0 &0\\
		0 & 1 & 0\\
		1 & 0 & 1
	\end{bmatrix} \cdot
	\begin{bmatrix}
		1 & 0 &0\\
		-2 & 1 & 0\\
		0 & 0 & 1
	\end{bmatrix} \cdot&
	\begin{bmatrix}
		2 & 1 & 1 & 5\\
		4 & -6 & 0 & -2\\
		-2 & 7 & 2 & 9
	\end{bmatrix} = 
	\begin{bmatrix}
		2 & 1 & 1 & 5\\
		0 & -8 & -2 & -12\\
		0 & 0 & 1 & 2
	\end{bmatrix}
\end{align*}

\subsection{Conclusão}
Com este material sabemos como encontrar a solução de um sistema linear utilizando a Eliminação Gaussiana e as operações Elementares, com suas respectivas matrizes. O próximo assunto a ser abordado será \textbf{Fatoração LU}.

\newpage


\section{Fatoração A=LU}

\subsection{Sem permutação de linhas}

No capítulo anterior vimos, ou relembramos,  como resolver um sistema linear utilizando o processo da Eliminação Gaussiana por meio, principalmente, de operações elementares e suas matrizes. Neste capítulo continuaremos estudando sistemas lineares do tipo $Ax=b$ e apresentaremos uma maneira de fatorar a matriz $A$, escrevendo-a como $A=LU$.\\

Dito isso, já podemos definir a matriz $A$ como a matriz de coeficientes do nosso sistema linear, ou seja, exatamente a mesma matriz $A$ do capítulo anterior. Nosso sistema linear é:\\
\begin{equation*}
	\begin{bmatrix}
		a_{11} & a_{12} & a_{13}\\
		a_{21} & a_{22} & a_{23}\\
		a_{31} & a_{32} & a_{33}
	\end{bmatrix} \cdot
	\begin{bmatrix}
		x\\ y\\z
	\end{bmatrix}=
	\begin{bmatrix}
		b_{11} \\ b_{21} \\ b_{31}
	\end{bmatrix} \mbox{ logo, } \newline
	A=\begin{bmatrix}
		a_{11} & a_{12} & a_{13}\\
		a_{21} & a_{22} & a_{23}\\
		a_{31} & a_{32} & a_{33}
	\end{bmatrix}
\end{equation*}\\

A matriz $U$ é a matriz triangular superior que aparece ao final do processo de escalonamento da matriz $A$, obtida por meio das operações elementares. Você deve se lembrar que, em nossa aula 2 de \textit{MATLAB}, aprendemos a função $[\mbox{~},U]=lu(A)$, sendo $U$ o nome dado à variável que guarda o output da função \textit{lu()}, isto é, a matriz escalonada resultante da eliminação gaussiana. Com $U$ em mãos, tudo que nos restava fazer era uma substituição retroativa para descobrir a solução do sistema. \\

A última matriz que falta ser descoberta é $L$. Para isso, precisamos lembrar das matrizes $E_i$ que representam as operações elementares. Se nos recordarmos, para escalonar $A$ até $U$ fazíamos:

\begin{equation*}
	U=E_n\cdot E_{n-1} \cdot ... \cdot E_2\cdot E_1\cdot A
\end{equation*}	
sendo $n$ o número de operações elementares a serem feitas. \\

Chamemos de $E$ a matriz resultante de todas as multiplicações de $E_i$. Podemos reescrever a equação acima como $U=E\cdot A$. Queremos chegar na faturação $A=LU$, logo, não é difícil perceber que basta multiplicar ambos os lados de $U=E\cdot A$ por $E^{-1}$ à esquerda que obteremos algo similar à fatoração desejada.
\begin{equation*}
	E^{-1}U=E^{-1}E\cdot A \rightarrow E^{-1}\cdot U=A
\end{equation*}\\

De fato, a matriz $L$ da fatoração $A=LU$ é, justamente, a multiplicação de todas as inversas das matrizes elementares utilizada. Sendo assim, definimos
\begin{equation*}
	L=E_1^{-1}\cdot E_2^{-1}\cdot ... \cdot E_n^{-1}
\end{equation*}

Convença-se de que $L$ está corretamente definida!

O único empecilho para esta definição é garantir que todas as matrizes elementares são inversíveis. Para isso, seus determinantes devem ser diferentes de 0. Como estamos estudando, nesta seção, apenas o caso sem trocas de linha, é trivial notar que todas as matrizes $E_i$ possuem 1 em sua diagonal principal e são triangulares inferiores. Sendo assim, seus determinantes são sempre 1. Convença-se deste fato.\\

Agora podemos apresentar a versão completa da função do \textit{MATLAB}, $[L,U]=lu(A)$. Esta função retorna, não somente a matriz escalonada $U$, como a matriz $L$, o que faz todo o sentido dado o nome da função... A análise da matriz $L$ é importante para ver se houve trocas de linha na execução interna do algoritmo da função.

Com todas estas definições em mãos, podemos partir para um exemplo.
\subsubsection{Exemplo}
Dada a seguinte matriz $A$, calculemos cada uma das matrizes envolvidas na fatoração $A=LU$ e mostremos que essa igualdade vale.

\begin{equation*}
	A=\begin{bmatrix}
		1 & 1 & 1\\
		2 & 3 & 5\\
		4 & 6 & 8
	\end{bmatrix} 
\end{equation*}\\

Realizando seu escalonamento, chegamos às seguintes matrizes elementares:

\begin{equation*}
	E_1=\begin{bmatrix}
		1 & 0 & 0\\
		0 & 1 & 0\\
		-4 & 0 & 1
	\end{bmatrix} ;
	E_2=
	\begin{bmatrix}
		1 & 0 & 0\\
		-2 & 1 & 0\\
		0 & 0 & 1
	\end{bmatrix};
	E_3=
	\begin{bmatrix}
		1 & 0 & 0\\
		0 & 1 & 0\\
		0 &-2 & 1
	\end{bmatrix}
\end{equation*}\\

Podemos verificar (deixo por conta de você, caro aluno) que:

\begin{align*}
	E_3\cdot E_2\cdot E_1\cdot A=U \mbox{ onde, }
	U=
	\begin{bmatrix}
		1 & 1 & 1\\
		0 & 1 & 3\\
		0 & 0 & -2
	\end{bmatrix}
\end{align*}	

Note que $U$ é uma matriz triangular superior assim como prevê a teoria! Calculemos agora a matriz $L$.

\begin{align}
	L=E_1^{-1}\cdot E_2^{-1}\cdot E_3^{-1}=
	\begin{bmatrix}
		1 & 0 & 0\\
		2 & 1 & 0\\
		4 & 2 & 1
	\end{bmatrix}
\end{align}

Como previsto, a matriz $L$ é triangular inferior com todas as entradas da diagonal principal igual a 1.

\textbf{Dica:} para inverter uma matriz elementar basta trocar o sinal da entrada não nula fora da diagonal principal. 

Podemos, por fim, verificar que:

\begin{equation*}
	L\cdot U=\begin{bmatrix}
		1 & 0 & 0\\
		2 & 1 & 0\\
		4 & 2 & 1
	\end{bmatrix}\cdot
	\begin{bmatrix}
		1 & 1 & 1\\
		0 & 1 & 3\\
		0 & 0 & -2
	\end{bmatrix}=
	\begin{bmatrix}
		1 & 1 & 1\\
		2 & 3 & 5\\
		4 & 6 & 8
	\end{bmatrix}=A
\end{equation*}

\subsection{Fatoração PA=LU (com permutação de linhas)}

Caso seja necessário realizar alguma permutação de linhas a fim de garantir que $U$ será uma matriz escalonada, precisamos corrigir a matriz $A$, introduzindo as permutações necessárias para, então, realizar a fatoração $LU$. Uma vez detectadas as permutações realizadas, podemos carregar essa informação em uma matriz $P$ e multiplicá-la por $A$ de modo que
\begin{equation*}
	PA=LU
\end{equation*}

Vejamos um exemplo.

\subsubsection{Exemplo}
Dada a seguinte matriz $A$, calculemos cada uma das matrizes envolvidas na fatoração $A=LU$, mostremos que serão necessárias permutações, montemos a matriz $P$ e verifiquemos a validade da igualdade $PA=LU$.

\begin{equation*}
	A=\begin{bmatrix}
		1 & 2 & 3\\
		2 & 4 & 5\\
		1 & 3 & 4
	\end{bmatrix}
\end{equation*}

Para simplificar as contas, usemos a função do \textit{MATLAB} $[L,U]=lu(A)$. O retorno desta função é:

\begin{equation*}
	L=\begin{bmatrix}
		0.5 & 0 & 1\\
		1 & 0 & 0\\
		0.5 & 1 & 0
	\end{bmatrix}; U=
	\begin{bmatrix}
		2 & 4 & 5\\
		0 & 1 & 1.5\\
		0 & 0 & 0.5
	\end{bmatrix}
\end{equation*}	

A matriz $L$, neste caso, não é triangular inferior, o que indica que o algoritmo interno da função realizou permutações na matriz $A$. Precisamos, então, montar a matriz $P$ de permutações. Podemos começar trocando as linhas 2 e 3. Para isso, chamemos de $P_1$ a seguinte matriz de modo que...

\begin{equation*}
	P_1=
	\begin{bmatrix}
		1 & 0 & 0\\
		0 & 0 & 1\\
		0 & 1 & 0
	\end{bmatrix} \mbox{ de modo que }
	P_1\cdot L=
	\begin{bmatrix}
		0.5 & 0 & 1\\
		0.5 & 1 & 0\\
		1 & 0 & 0
	\end{bmatrix}
\end{equation*}

Agora definimos $P_2$ a partir da troca das linhas 1 e 3.
\begin{equation*}
	P_2=
	\begin{bmatrix}
		0 & 0 & 1\\
		0 & 1 & 0\\
		1 & 0 & 0
	\end{bmatrix} \mbox{ de modo que }
	P_2\cdot P_1\cdot L=
	\begin{bmatrix}
		1 & 0 & 0\\
		0.5 & 1 & 0\\
		0.5 & 0 & 1
	\end{bmatrix}
\end{equation*}

Definimos, então, $P=P_2\cdot P_1$. O valor da matriz $P$ está exibido logo abaixo.

Agora a matriz L possui as características necessárias segundo a teoria, isto é, ser triangular inferior e possuir todas as entradas da diagonal principal igual a 1.

Podemos, finalmente, verificar que:
\begin{equation*}
	PA=LU \leftrightarrow 
	\begin{bmatrix}
		0 & 1 & 0\\
		0 & 0 & 1\\
		1 & 0 & 0
	\end{bmatrix}
	\cdot  
	\begin{bmatrix}
		1 & 2 & 3\\
		2 & 4 & 5\\
		1 & 3 & 4
	\end{bmatrix}=
	\begin{bmatrix}
		0.5 & 0 & 1\\
		1 & 0 & 0\\
		0.5 & 1 & 0
	\end{bmatrix} \cdot
	\begin{bmatrix}
		2 & 4 & 5\\
		0 & 1 & 1.5\\
		0 & 0 & 0.5
	\end{bmatrix}
\end{equation*}

\subsection{Conclusão}
Neste capítulo foram apresentados os conceitos de fatoração $A=LU$ em seu caso sem permutação de linhas e $PA=LU$ quando são necessárias essas permutações. Este é um algoritmo importante para a compreensão dos métodos de resolução de sistemas lineares e seu entendimento desmistifica o funcionamento da função $lu()$ utilizada no \textit{MATLAB}.

	\section{Espaços Fundamentais de uma Matriz}
\subsection{Definições:}

Estudaremos os 4 subespaços fundamentais de uma matriz. Para todo este estudo, considere A uma matriz $m\times n$ São eles:
\begin{enumerate}
	\item Espaço Coluna, ou Imagem
	\item Espaço Linha
	\item Espaço Nulo, ou Núcleo
	\item Espaço Numo da transposta
\end{enumerate}

\subsubsection{Espaço Coluna - $Im(A)$}
O espaço coluna, ou imagem de uma matriz $A$ é o subespaço vetorial gerado pelas colunas da matriz $A$.\\

\textbf{Def:}
\begin{equation*}
	Im(A)=\{v \in R^m \mbox{ tal que } A\cdot u=v \mbox{ para algum }u\in R^n\}
\end{equation*}

É importante lembrar de alguns conceitos como \textit{Espaços e Subespaços Vetoriais} e \textit{Independência Linear}, uma vez que nada garante que as $m$ colunas sejam $LI$ e gerem um espaço de dimensão $m$.\\

\textbf{Base para $Im(A)$:} podemos fazer uma Eliminação Gaussiana de $A$ e observar quais colunas da matriz $U$ resultante deste processo possuem pivôs. Se as colunas $c_i$ de $U$ possuem pivô, então as colunas $c_i$ de $A$ serão base da \textit{Imagem} de $A$. Consegue perceber o por quê?\\

\textbf{Observação Importante:} $Im(A)\neq Im(U)$\\

\textbf{Posto} de uma matriz $A$ é a dimensão da \textit{Imagem} dessa matriz $A$ $\rightarrow posto(A)=dim(Im(A))$

\subsubsection{Espaço Nulo - $N(A)$}

O espaço nulo de $A$ é o espaço vetorial gerado pelos vetores $x$ tal que $A\cdot x=0$.\\

\textbf{Def:}
\begin{equation*}
	N(A)=\{x \in R^n \mbox{ tal que }A\cdot x=0\}
\end{equation*}

\subsubsection{Espaço Linha - $Im(A^T)$}

O espaço linha de $A$ é o espaço vetorial gerado pelos vetores linha de $A$. De modo análogo, é o\textit{ espaço coluna da matriz transposta} de $A$.\\

\textbf{Def:}
\begin{equation*}
	Im(A^T)=\{v \in R^m \mbox{ tal que } A\cdot u=v \mbox{ para algum }u\in R^n\}
\end{equation*}

Outra maneira de encontrar o espaço linha de $A$ é, novamente, por meio da Eliminação Gaussiana. Note que, se $U$ for a matriz escalonada da Eliminação Gaussiana, então o espaço linha de $U$ é igual ao espaço linha de $U$. Isso quer dizer que uma base de $Im(U^T)$ é também base de $Im(A^t)$.

\subsubsection{Espaço Nulo da Transposta- $N(A^T)$}

O espaço nulo da transposta de $A$ é o espaço vetorial gerado pelos vetores $x$ tal que $A^T\cdot x=0$.\\

\textbf{Def:}
\begin{equation*}
	N(A^T)=\{x \in R^m \mbox{ tal que }A^T\cdot x=0\}
\end{equation*}


\subsection{Exemplo:}

Encontre os 4 espaços fundamentais da matriz abaixo.
\begin{equation*}
	A= 
	\begin{bmatrix}
		1 & 1 & 1\\
		2 & 2 & 2\\
		3 & 3 & 3
	\end{bmatrix}	
\end{equation*}\\

Para encontrar o espaço coluna de $A$, vamos escalonar esta matriz. Podemos usar o comando já aprendido $[,U]=lu(A)$, que nos retorna:
\begin{equation*}
	U= 
	\begin{bmatrix}
		3 & 3 & 3\\
		0 & 0 & 0\\
		0 & 0 & 0
	\end{bmatrix}	
\end{equation*}\\

Já podemos perceber a existência de apenas 1 pivô, logo $\textbf{posto(A)=dim(Im(A))=1}$. Com além disso, como o pivô está na primeira coluna de $U$, a base da imagem de $A$ será formada pela primeira coluna de $A$. Também podemos usar a função \textit{colspace(sym((A))} do MATLAB.
\begin{equation*}
	\beta_{Im(A)}= 
	\begin{pmatrix}
		1\\
		2\\
		3
	\end{pmatrix}	
\end{equation*}\\

Para o núcleo de $A$ devemos resolver o sistema linear $A\cdot x=0$ e os vetores $x$ que satisfizerem esta igualdade serão nosso núcleo. Analogamente, e para facilitar nossa vida, podemos usar o comando \textit{B=null(sym(A))}, que retorna:
\begin{equation*}
	B=
	\begin{pmatrix}
		-1 & -1\\
		1 & 0\\
		0 & 1
	\end{pmatrix}	
\end{equation*}\\

Podemos confirmar esta resposta multiplicando $A*B$ e verificando que esta conta dá \textbf{zero}. \\

\textbf{Base:} para verificar que estes vetores nas colunas de $B$ são base do núcleo, devemos verificar que eles são \textit{LI}. Uma vez confirmado, temos que:

\begin{equation*}
	\beta_{N(A)}= \{
	\begin{pmatrix}
		-1\\
		1\\
		0
	\end{pmatrix},
	\begin{pmatrix}
		-1\\
		0\\
		1
	\end{pmatrix}	\}
\end{equation*}\\

\textbf{Observação:} note que $dim(N(A))=2$. Isso faz sentido pois, lembrando de Álgebra 1, $dim(Im(A))+dim(N(A))=n$.

Para calcularmos o espaço linha, temos duas opções. Primeiro, transpor a matriz $A$ e calcular a imagem desta nova matriz da maneira já explicada. Por exemplo: \textit{colspace(sym(transpose(A)))}. Outra maneira é realizar a fatoração $LU$ e olhar para as linhas de $U$, uma vez que $Im(U^T)=Im(A^T)$. Como já temos o resultado da função $lu(A)$, podemos notar que a primeira linha de $U$ é base do espaço linha de $U$. Logo,

\begin{equation*}
	\beta_{Im(A^T)}= 
	\begin{pmatrix}
		3\\
		3\\
		3
	\end{pmatrix}	
\end{equation*}\\

Finalmente, para o espaço nulo da transposta, podemos utilizar um processo similar ao cálculo do espaço nulo. Será que \textit{null(sym(transpose(A)))}, que retorna:

\begin{equation*}
	\begin{pmatrix}
		-2 & -3\\
		1 & 0\\
		0 & 1
	\end{pmatrix}	
\end{equation*}\\

Verificamos se estes vetores são, de fato, a base do núcleo da transposta ao verificar que eles satisfazem $transpose(A)\cdot null(symtranspose(A)))=0$ e que eles são $LI$. Por fim, temos
\begin{equation*}
	\beta_{N(A)}= \{
	\begin{pmatrix}
		-2\\
		1\\
		0
	\end{pmatrix},
	\begin{pmatrix}
		-3\\
		0\\
		1
	\end{pmatrix}	\}
\end{equation*}\\

\subsection{Conclusão}
Neste capítulo foram apresentados os 4 espaços fundamentais de uma matriz qualquer, bem como os procedimentos necessários para calcular estes subespaços vetoriais. Na próxima aula veremos relações de ortogonalidade entre estes subespaços.
\end{document}

